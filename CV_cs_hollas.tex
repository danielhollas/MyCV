\documentclass[a4paper,10pt]{article}
\usepackage{array, xcolor, lipsum, bibentry}
\usepackage[margin=1.5cm,total={17cm,25cm}, top=1cm, left=2.5cm, includefoot]{geometry}

	 
\title{\bfseries\huge Curriculum vitae}
\author{Daniel Hollas}
\date{28. b\v{r}ezna 2013}
 
\definecolor{lightgray}{gray}{0.8}
\newcolumntype{L}{>{\raggedleft}p{0.14\textwidth}}
\newcolumntype{R}{p{0.8\textwidth}}
\newcommand\VRule{\color{lightgray}\vrule width 0.5pt}


\begin{document}
\maketitle
\thispagestyle{empty}
%\vspace{0.5em}
\section*{Osobn\'{i} \'{u}daje}
%\begin{minipage}[ht]{0.48\textwidth}
\begin{tabular}{L!{\VRule}R}
Jm\'{e}no %a p{r}\'{i}jmen\'{i} 
& Bc.\,Daniel Hollas \\
Adresa&Ol\v{s}ovec 50\\
&753 01 Hranice\\
Datum nar. %ozen\'{i} 
& 10.6.1989 \\
N\'{a}rodnost & \v{c}esk\'{a} \\
%Rodinn\'{y} stav& svobodn\'{y} \\
Mobiln\'{i} telefon &+420 737 876 686 \\
E-mail & hollasd@vscht.cz \\
%\end{minipage}
\end{tabular}
%\vspace{15pt}

\section*{Vzd\v{e}l\'{a}n\'{i}}
\begin{tabular}{L!{\VRule}R}
2011--nyn\'{i}&\textbf{magistersk\'{e} studium, obor Fyzik\'{a}ln\'{i} chemie}, \textit{Vysok\'{a} \v{s}kola chemicko-technologick\'{a} v Praze}, Praha. \\
& Diplomov\'{a} pr\'{a}ce: Simulace kvanto\'{y}ch efekt\r{u} jader v termodynamice a spektroskopii \\
	& \v{S}kolitel: doc.\,RNDr.\,Petr\,Slav\'{i}\v{c}ek,\,Ph.D.\vspace{5pt}\\
2008--2011&\textbf{bakal\'{a}\v{r}sk\'{e} studium, obor Chemie}, \textit{Vysok\'{a} \v{s}kola chemicko-technologick\'{a} v Praze}, Praha, \textit{cum laude}. \\
	& Bakal\'{a}\v{r}sk\'{a} pr\'{a}ce: Fotochemie t\v{e}\v{z}k\'{y}ch analog\r{u} cyklopropenu \\
	& \v{S}kolitel: doc.\,RNDr.\,Petr\,Slav\'{i}\v{c}ek,\,Ph.D.\vspace{5pt}\\
2000--2008&\textbf{Gymn\'{a}zium}, \textit{Gymn\'{a}zium Hranice}, Hranice, \textit{cum laude}.\\
\end{tabular}

%\section*{Summer schools and conferences}
%\begin{tabular}{L!{\VRule}R}
%10/2012&Winter school, \textit{GPU Computing - Methods and Applications in the Natural Sciences}, Wroclaw - Poland.\\
%7/2012 &The 12th Sostrup Summer School, \textit{Quantum Chemistry and Molecular Properties}, Himmelbjerget - Denmark  \\
%7/2012 &\textit{XXIV IUPAC Symposium on Photochemistry}, Coimbra-Portugal  \\
%7/2011 & CUSO Summer School 2011, \textit{Solving the Schr\"{o}dinger Equation - From Electronic Structure to Quantum Dynamics}, Villars sur Ollon - Switzerland \\
%9/2009 & \textit{Central European Symposium on Theoretical Chemistry}, Dobog\'{o}k\H{o} - Hungary \\
%\end{tabular}

\section*{Jazykov\'{e} dovednosti}
\begin{tabular}{L!{\VRule}R}
anglick\'{y}&velmi dobr\'{a} znalost (First Certificate in English, grade A)\\
n\v{e}meck\'{y}&za\v{c}\'{a}te\v{c}n\'{i}k\\
\end{tabular}

\section*{Po\v{c}\'{i}ta\v{c}ov\'{e} dovednosti}
\begin{tabular}{L!{\VRule}R}
OS			&\textsc{Linux}, \textsc{Microsoft Windows}\\
dokumenty 	&\LaTeX, \textsc{Microsoft Office}\\
programov\'{a}n\'{i} & C++, Fortran, BASH, awk \\
v\v{e}deck\'{e}		&kvantov\'{a} chemie (\textsc{Gaussian}, \textsc{MOLPRO})\\
			&molekul\'{a}rn\'{i} dynamika (\textsc{AMBER}, \textsc{GROMACS}) \\
&vizualizace molekul (\textsc{VMD}, \textsc{PyMOL})\\
&algebraick\'{e} syst\'{e}my (\textsc{Maple}, \textsc{Mathematica}) \\
\end{tabular}	 	 
	 
%\section*{Publikace}
%\begin{tabular}{L!{\VRule}R}
%2012&Hollas D., Svoboda O., Chmela J., Slav\'{i}\v{c}ek P., 
%FOTODYNAMIKA VODY VYVOLANÁ IONIZUJÍCÍM ZÁŘENÍM: MOLEKULÁRNÍ POHLED
%\textit{Chem. Listy}, \textbf{2012}, \textit{106}, 936.\\ 

%2010&Ban\'{a}\v{s}, P., Hollas, D., Zgarbov\'{a}, M., Jure\v{c}ka, P., Orozco, M., Cheatham, T. E., \v{S}poner, J., Otyepka M., 
%Performance of Molecular Mechanics Force Fields for RNA Simulations: Stability of UUCG and GNRA Hairpins
%\textit{J.~Chem. Theory Comput.}, \textbf{2010}, \textit{6}(12), 3836–3849. \\

%2010&Ml\'{y}nsk\'{y}, V., Ban\'{a}\v{s}, P., Hollas, D., R\'{e}blov\'{a}, K., Walter, N. G., \v{S}poner, J., Otyepka, M. 
%Extensive molecular dynamics simulations showing that canonical G8 and protonated A38H+ forms are most consistent with crystal structures of hairpin ribozyme.
%\textit{J.~Phys. Chem. B}, \textbf{2010}, \textit{114}(19), 6642–52\\
%\end{tabular}

 
\end{document}
